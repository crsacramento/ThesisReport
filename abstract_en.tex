\chapter*{Abstract}
Graphical user interfaces (\textit{GUIs}) are populated with recurring behaviors that vary only slightly. For example, authentication (login / password) is a behavior common to many software applications. However, there are different behaviors between different implementations of this behavior. Sometimes a message appears when the user does not enter the correct data, sometimes, the application software only erases entered data and shows no indication to the user. These recurring behaviors (\textit{UI patterns}) are well identified in the literature.

The goal of this dissertation is to continue the work already done on an existing tool called PARADIGM-RE, a dynamic reverse engineering approach to extract User Interface (UI) Patterns from existent Web applications. As such, we will develop a data analysis module with the goal of improving and substantiate the existing identifying heuristics set, and we will extend the current set of identifiable patterns.

First the theme to be developed during the course of the dissertation is introduced, starting by defining the context and issue at hand and describing the goals of this dissertation. Afterwards we present a literary review on reverse engineering approaches, approaches that infer patterns from Web applications, and data mining algorithms and tools relevant to the problem. Lastly, we will provide an estimated work plan for the project development.

\chapter*{Resumo}
\begin{otherlanguage}{portuguese}
As interfaces gráficas estão populadas de comportamentos recorrentes que variam apenas ligeiramente. Por exemplo, a autenticação (\textit{login/password}) é um comportamento comum a muitas aplicações de software. No entanto, há comportamentos diferentes entre diferentes implementações desse padrão. Por vezes, aparece uma mensagem quando o utilizador não introduz os dados correctos, outras vezes, a aplicação de software apenas apaga os dados introduzidos e não apresenta indicação nenhuma ao utilizador. Estes comportamentos recorrentes (\textit{padrões de interface}) estão bem identificados na literatura. 

O objetivo deste trabalho é dar continuidade ao trabalho já realizado numa ferramenta existente chamada PARADIGM-RE, uma abordagem de engenharia reversa dinâmica para extrair padrões de interface de aplicações Web existentes. Como tal, vamos desenvolver um módulo de análise de dados, cujo objetivo é melhorar e fundamentar as heurísticas de identificação existentes definidas, e vamos ampliar o atual conjunto de padrões identificáveis.

Primeiro é introduzido o tema a ser desenvolvido durante da dissertação, a começar por definir o contexto e o assunto em questão e descrever os objetivos desta dissertação. Em seguida apresentamos uma revisão literária sobre abordagens de engenharia reversa, abordagens que inferem padrões de aplicações Web, e os algoritmos e ferramentas relevantes para o problema de análise de dados. Por fim, iremos fornecer um plano de trabalho estimado para o desenvolvimento do projeto.
\end{otherlanguage}
