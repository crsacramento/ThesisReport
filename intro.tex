\chapter{Introduction} \label{chap:intro}

%\section*{}
GUIs \textit{(Graphical User Interfaces)} of all kinds are populated with recurring behaviours that vary slightly. For example, authentication \textit{(login/password)} is a common behaviour in many software applications. However, the implementation of those behaviours may vary significantly. For a login, in some cases an error message may appear when the authentication fails; in others, the software application simply erases the inserted data and doesn't send a message to the user. These behaviours (patterns) are called User Interface (UI) patterns \cite{van2001patterns} and are recurring solutions that solve common design problems. 



\section{Context and Motivation} \label{sec:context}


\section{Problem Description} \label{sec:problemdescription}


\section{Goals} \label{sec:goals}
Considering the problem described and the proposed solution, the main goal for this research work is

%\begin{itemize}
%
%  \item
%
%  \item
%
%\end{itemize}
%
%

\section{Structure of the Report} \label{sec:outline}

Besides the introduction chapter, this document is composed three additional chapters. These chapters have the following structure:

\begin{description}

  \item[Chapter \ref{chap:sota}]
  introduces essential concepts to understand the problems with which this document deals. Furthermore, we describe the [cenas]. Lastly, we give some insight about data mining algorithms and how they will be applied to this work.

  \item[Chapter \ref{chap:workplan}]
  outlines the main steps in the development of this thesis (and the respective
  software prototype) and attempts to provide a feasible schedule for the work's
  execution.

  \item[Chapter \ref{chap:conclusions}]
  sums up the what has been defined in the report, emphasizing the problem that
  the thesis addresses and the work that will be executed towards solving that
  problem. It will also give a brief idea of what are the expected results at
  the end of the project.

\end{description}
